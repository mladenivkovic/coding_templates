\documentclass[11pt]{beamer}


\usepackage{lmodern} 		% Diese beiden packages sorgen für echte 
\usepackage[T1]{fontenc}	% Umlaute.

\usepackage{amssymb, amsmath, color, graphicx, float, setspace, tipa}
\usepackage[utf8]{inputenc} 
\usepackage[english]{babel}
\usepackage[justification=centering]{caption}
\addto\captionsenglish{\renewcommand{\figurename}{}} % Redefine how figures are called


%\usepackage[pdfpagelabels,pdfstartview = FitH,bookmarksopen = true,bookmarksnumbered = true,linkcolor = black,plainpages = false,hypertexnames = false,citecolor = black, breaklinks]{hyperref}
%\usepackage{url}
\usepackage{longtable} 		%Seitenübergreifende Tabelle. Vorlage siehe unten
\newtheorem*{bem}{Bemerkung} % Neue Theorem-Umgebung: Bemerkung
\newcommand{\fillframe}{\vskip0pt plus 1filll} 


% CAPTIONS
\usepackage[]{caption}
\addto\captionsenglish{\renewcommand{\figurename}{}} %Abbildungen nicht bzw. anders beschriften.
\captionsetup{font={small,sf}, labelfont=bf, format=plain}
% Use captionsetup below for no 'Figure' label
%\captionsetup{font={small,tt}, labelformat=empty, format=plain}



%===================
%BIBLIOGRAPHY
%===================
\usepackage[backend=bibtex,sorting=nyt,bibencoding=ascii,citestyle=authoryear]{biblatex}
\addbibresource{references.bib}
\usepackage{csquotes} %recommended when using babel and biblatex
%Abbreviations for Bibliography: http://adsabs.harvard.edu/abs_doc/aas_macros.html







%-----------------
% BEAMER-SPECIFIC
%-----------------

\usetheme{metropolis}
% deactivate a new page when a new section begins
\metroset{sectionpage=none} 
\usepackage{FiraSans}
\usefonttheme[onlymath]{serif}


% You can look at different options for usetheme, usecolortheme und usefonttheme 
% here: http://deic.uab.es/~iblanes/beamer_gallery/


% transparent Overlays (what's coming next on the slide is shown with reduced alpha)
%\setbeamercovered{transparent} 

% Removes navigation symbols at the bottom
%\beamertemplatenavigationsymbolsempty 

% Use page numbers as links
%\setbeamertemplate{footline}[frame]  
%    \setbeamertemplate{footline}{%
%    	\raisebox{5pt}{\makebox[\paperwidth]{\hfill\makebox[10pt]{\hyperlink{tableofcontents}{\scriptsize\insertframenumber}}}}}




%==============================================
% This file contains my definitions and
% newcommands. Makes things easy to copypaste
% between projects.
%==============================================


%--------------------------------------------
% Math Stuff
%--------------------------------------------

\newcommand{\corresponds}{\mathrel{\widehat{=}}}       % equals with hat

\newcommand {\arctanh}{\mathrm{arctanh}}               % Atanh
\newcommand{\arccot}{\mathrm{arccot }}                 % Acotanh

\newcommand{\e}{\mbox{e}}                              % e noncursive in math mode

\newcommand{\del}{\partial}                            % partial diff operator
\newcommand{\de}{\mathrm{d}}                           % differential d
\newcommand{\D}{\mathrm{d}}                            % differential d
\newcommand{\GRAD}{\mathrm{grad}\ }                    % gradient
\newcommand{\DIV}{\mathrm{div}\ }                      % divergence
\newcommand{\ROT}{\mathrm{rot}\ }                      % rotation

\newcommand{\CONST}{\mathrm{const.\ }}                 % constant
\newcommand{\var}{\mathrm{var}}                        % variance

\newcommand{\msol}{M_\odot}                            % solar mass
\newcommand{\order}{\mathcal{O}}                       % order, e.g. O(h^2)

\newcommand{\dete}{\mathrm{d}t}                        % dt
\newcommand{\delte}{\del t}                            % partial t
\newcommand{\dex}{\mathrm{d}x}                         % dx
\newcommand{\delx}{\del x}                             % partial x
\newcommand{\der}{\mathrm{d}r}                         % dr
\newcommand{\delr}{\del r}                             % partial r


\newcommand{\deldx}{\frac{\del}{\del x}}				% shortcut partial derivative, in line
\newcommand{\ddx}{\frac{\de}{\de x}}					% shortcut total derivative, in line
\newcommand{\DELDX}[1]{\frac{\del  #1}{\del x}}			% shortcut partial derivative, on fraction
\newcommand{\DDX}[1]{\frac{\de  #1}{\de x}}				% shortcut total derivative, on fraction

\newcommand{\deldvecx}{\frac{\del}{\del \x}}	   		% shortcut partial derivative, in line
\newcommand{\ddvecx}{\frac{\de}{\de \x}}				% shortcut total derivative, in line
\newcommand{\DELDVECX}[1]{\frac{\del  #1}{\del \x}}		% shortcut partial derivative, on fraction
\newcommand{\DDVECX}[1]{\frac{\de  #1}{\de \x}}			% shortcut total derivative, on fraction

\newcommand{\deldr}{\frac{\del}{\del r}}				% shortcut partial derivative, in line
\newcommand{\ddr}{\frac{\de}{\de r}}					% shortcut total derivative, in line
\newcommand{\DELDR}[1]{\frac{\del  #1}{\del r}}			% shortcut partial derivative, on fraction
\newcommand{\DDR}[1]{\frac{\de  #1}{\de r}}				% shortcut total derivative, on fraction


%\usepackage{mathtools}

\usepackage{pgfpages}
% These slides also contain speaker notes. You can print just the slides,
% just the notes, or both, depending on the setting below. Comment out the want
% you want.

%\setbeameroption{hide notes} % Only slides
%\setbeameroption{show only notes} % Only notes
\setbeameroption{show notes on second screen=right}




%---------------------
%  Metadata
%---------------------

\title[GEAR-RT]{GEAR-RT - Radiative Transfer with SWIFT}
%     Title of the presentations
\subtitle[]{Radiation Hydrodynamics with Meshless Methods, the M1 Closure, Interdependent Tasking, and Sub-Cycling with Individual Timestepping}
%     Subtitle
\author[M. Ivkovic]{Mladen Ivkovic}
%     Set author
\institute[LASTRO EPFL]{
%    \begin{columns}
%        \column{0.7\textwidth}
            LASTRO \\
            \'Ecole Polytechnique F\'ed\'erale de Lausanne
%        \column{0.2\textwidth}
%            \makebox[\textwidth][r]{\includegraphics[width=2cm]{figures/epfl-logo.pdf}}
%    \end{columns}
}




%     Institute
\date[12.08.2022]{12. August 2022}
%	  Date of presentation. Alternatively you can use \today to set compilation time date automatically.
% \logo{\pgfimage[width=2cm,height=2cm]{mylogo}}
%     Add mylogo.pdf (bzw. mylogo.png, mylogo.jpg, mylogo.mps if you're using pdftex as Backend) as the logo on all slides, here using the package pgf.
% \titlegraphic{\includegraphics[width=2cm,height=2cm]{mylogo}}
%	  Add mylogo.pdf/png/jpg file as the logo only on the title packe using the graphicx package








%===============================================================================%===============================================================================%===============================================================================



\begin{document}


\begin{frame}{}
	\titlepage
	\note{testing notes}
\end{frame}


%\begin{frame}{Outline}\label{tableofcontents}
%   \tableofcontents
%\end{frame}





%---------------------------------------------------------
\begin{frame}{Equations of Radiative Transfer}
%---------------------------------------------------------
	Can we assume $\DELDT{I_\nu} \approx 0$ ?

	Yes, if we are primarily focussing on the \textbf{fluid flow} as opposed to the \textbf{radiation flow}.

	\begin{itemize}
	\item Optically thin regime:

		$t_{fluid} \sim l / v$

		$t_{rad} \sim l / c$

		$\Rightarrow$ $t_{rad} / t_{fluid} = \mathcal{O}(v/c)$, $t_{rad} \ll t_{fluid}$
	\end{itemize}

	So the radiation field has ample time to adjust itself to the changes induced by the fluid.
\end{frame}




%---------------------------------------------------------
\begin{frame}{Optical Depth}
%---------------------------------------------------------

	A few more words on $\tau_\nu$:

	If $\x$ and $\x'$ are two points in the medium separated by $l = |\x' - \x|$, the optical depth \textbf{between them} is
	%
	\begin{align*}
	\tau_\nu(\x, \x') = \int_0^l \alpha_\nu(\x + \mathbf{n}s; \mathbf{n}, \nu) \de s
	\end{align*}

	$	\tau_\nu(\x, \x')$ is equal to the number of mean free paths  between $\x$ and $\x'$
\end{frame}









%---------------------------------------------------------
\section{Monte Carlo Radiative Transfer Method}
%---------------------------------------------------------

\begin{frame}{Monte Carlo Radiative Transfer Method}

We have a nice equation for Radiative Transfer along the ray - so let's use it!

Each radiation source emits \emph{photon packets} with

\begin{itemize}
	\item some direction $\theta$, $\phi$
	\item some energy / photon number
	\item some frequency
\end{itemize}

Follow the packets, and solve the RT equation along the path of the ray.


\end{frame}





%---------------------------------------------------------
\begin{frame}{Final Remarks}
%---------------------------------------------------------

\usebeamertemplate*{title separator}

\end{frame}






%-------------------------------------------------
\begin{frame}{Photon Packet Path Length}
%-------------------------------------------------
	\begin{columns}
		\column{0.5\linewidth}{

		   \begin{align*}
   		   \de I_\nu &= -I_\nu\ n\ \sigma\ \de l \\
   		   \Rightarrow \quad I_\nu &= \exp(-n \sigma l)
		   \end{align*}

		   \begin{itemize}
		   \item $n \sigma$ is the fraction absorbed or scattered per length.

		   \item $n \sigma \de l$ is also the probability of interaction over $\de l$. Therefore, the probability to \emph{not} interact is $(1 - n \sigma \de l)$.
		   \end{itemize}
		 }
		\column{0.5\linewidth}{
		\begin{figure}
		\includegraphics[width=\linewidth]{figures/tikz/absorption.pdf}%
		\end{figure}
	}

	\end{columns}

\end{frame}






%\begin{frame}{References}
%    \cite{AHF}
%
%    \renewcommand*{\bibfont}{\footnotesize}
%    \bibliography{references}
%\end{frame}


\end{document}
