\documentclass[11pt]{beamer}


\usepackage{lmodern} 		% Diese beiden packages sorgen für echte 
\usepackage[T1]{fontenc}	% Umlaute.

\usepackage{amssymb, amsmath, color, graphicx, float, setspace, tipa}
\usepackage[utf8]{inputenc} 
\usepackage[english]{babel}
\usepackage[justification=centering]{caption}
\addto\captionsenglish{\renewcommand{\figurename}{}} % Redefine how figures are called


%\usepackage[pdfpagelabels,pdfstartview = FitH,bookmarksopen = true,bookmarksnumbered = true,linkcolor = black,plainpages = false,hypertexnames = false,citecolor = black, breaklinks]{hyperref}
%\usepackage{url}
\usepackage{longtable} 		%Seitenübergreifende Tabelle. Vorlage siehe unten
\newtheorem*{bem}{Bemerkung} % Neue Theorem-Umgebung: Bemerkung
\newcommand{\fillframe}{\vskip0pt plus 1filll} 


% CAPTIONS
\usepackage[]{caption}
\addto\captionsenglish{\renewcommand{\figurename}{}} %Abbildungen nicht bzw. anders beschriften.
\captionsetup{font={small,sf}, labelfont=bf, format=plain}
% Use captionsetup below for no 'Figure' label
%\captionsetup{font={small,tt}, labelformat=empty, format=plain}



%===================
%BIBLIOGRAPHY
%===================
\usepackage[backend=bibtex,sorting=nyt,bibencoding=ascii,citestyle=authoryear]{biblatex}
\addbibresource{references.bib}
\usepackage{csquotes} %recommended when using babel and biblatex
%Abbreviations for Bibliography: http://adsabs.harvard.edu/abs_doc/aas_macros.html







%-----------------
% BEAMER-SPECIFIC
%-----------------

\usetheme{metropolis}
% deactivate a new page when a new section begins
\metroset{sectionpage=none} 
\usepackage{FiraSans}
\usefonttheme[onlymath]{serif}


% You can look at different options for usetheme, usecolortheme und usefonttheme 
% here: http://deic.uab.es/~iblanes/beamer_gallery/


% transparent Overlays (what's coming next on the slide is shown with reduced alpha)
%\setbeamercovered{transparent} 

% Removes navigation symbols at the bottom
%\beamertemplatenavigationsymbolsempty 

% Use page numbers as links
%\setbeamertemplate{footline}[frame]  
%    \setbeamertemplate{footline}{%
%    	\raisebox{5pt}{\makebox[\paperwidth]{\hfill\makebox[10pt]{\hyperlink{tableofcontents}{\scriptsize\insertframenumber}}}}}




%==============================================
% This file contains my definitions and
% newcommands. Makes things easy to copypaste
% between projects.
%==============================================


%--------------------------------------------
% Math Stuff
%--------------------------------------------

\newcommand{\corresponds}{\mathrel{\widehat{=}}}       % equals with hat

\newcommand {\arctanh}{\mathrm{arctanh}}               % Atanh
\newcommand{\arccot}{\mathrm{arccot }}                 % Acotanh

\newcommand{\e}{\mbox{e}}                              % e noncursive in math mode

\newcommand{\del}{\partial}                            % partial diff operator
\newcommand{\de}{\mathrm{d}}                           % differential d
\newcommand{\D}{\mathrm{d}}                            % differential d
\newcommand{\GRAD}{\mathrm{grad}\ }                    % gradient
\newcommand{\DIV}{\mathrm{div}\ }                      % divergence
\newcommand{\ROT}{\mathrm{rot}\ }                      % rotation

\newcommand{\CONST}{\mathrm{const.\ }}                 % constant
\newcommand{\var}{\mathrm{var}}                        % variance

\newcommand{\msol}{M_\odot}                            % solar mass
\newcommand{\order}{\mathcal{O}}                       % order, e.g. O(h^2)

\newcommand{\dete}{\mathrm{d}t}                        % dt
\newcommand{\delte}{\del t}                            % partial t
\newcommand{\dex}{\mathrm{d}x}                         % dx
\newcommand{\delx}{\del x}                             % partial x
\newcommand{\der}{\mathrm{d}r}                         % dr
\newcommand{\delr}{\del r}                             % partial r


\newcommand{\deldx}{\frac{\del}{\del x}}				% shortcut partial derivative, in line
\newcommand{\ddx}{\frac{\de}{\de x}}					% shortcut total derivative, in line
\newcommand{\DELDX}[1]{\frac{\del  #1}{\del x}}			% shortcut partial derivative, on fraction
\newcommand{\DDX}[1]{\frac{\de  #1}{\de x}}				% shortcut total derivative, on fraction

\newcommand{\deldvecx}{\frac{\del}{\del \x}}	   		% shortcut partial derivative, in line
\newcommand{\ddvecx}{\frac{\de}{\de \x}}				% shortcut total derivative, in line
\newcommand{\DELDVECX}[1]{\frac{\del  #1}{\del \x}}		% shortcut partial derivative, on fraction
\newcommand{\DDVECX}[1]{\frac{\de  #1}{\de \x}}			% shortcut total derivative, on fraction

\newcommand{\deldr}{\frac{\del}{\del r}}				% shortcut partial derivative, in line
\newcommand{\ddr}{\frac{\de}{\de r}}					% shortcut total derivative, in line
\newcommand{\DELDR}[1]{\frac{\del  #1}{\del r}}			% shortcut partial derivative, on fraction
\newcommand{\DDR}[1]{\frac{\de  #1}{\de r}}				% shortcut total derivative, on fraction





%---------------------
%  Metadata
%---------------------

 \title[shortform]{Title}
%     Titel of the presentations
 \subtitle[shortform-subtitle]{Subtitile}
%     Subtitle
 \author[M. Ivkovic]{Mladen Ivkovic}
%     Set author
 \institute[EPFL]{LASTRO \\ \'Ecole Polytechnique F\'ed\'erale de Lausanne}
%     Institute
 \date[17.09.19]{17. September 2019}
%	  Date of presentation. Alternatively you can use \today to set compilation time date automatically.
% \logo{\pgfimage[width=2cm,height=2cm]{mylogo}}
%     Add mylogo.pdf (bzw. mylogo.png, mylogo.jpg, mylogo.mps if you're using pdftex as Backend) as the logo on all slides, here using the package pgf.
% \titlegraphic{\includegraphics[width=2cm,height=2cm]{mylogo}}
%	  Add mylogo.pdf/png/jpg file as the logo only on the title packe using the graphicx package









%===================================================================================
%===================================================================================
%
%
%
% \begin{frame}[Overlay-actrions][options]{Title}{Subtitle}
% 
%Overlay-actions
%	Overlay-actions set the default overlay actions of all environments within the 
%	frame, which allow action-specifications, e.g. \item in lists and blocks.
%	Overlays are 'animations' for elements on a slide, e.g. items appearing one after 
%	the other.
%
% 
%     <+->
%			makes the elements appear one by one.
% 
% options
% 
%     allowdisplaybreaks
%			allows a page break in multi-line equations. Works only together with option
%			allowframebreaks
%
%     allowframebreaks
%			automatically distribute the contents on multiple slides if it doesn't fit
%			on only one. However this disables overlays.
%
%     b,c,t
%			aligns frame to (b)ottom, (t)op or (c)enter
%
%     fragile
%			necessary for sourse text environments, e.g. verbatim
%
%     label=name
%			set a name for the frame so you can call it again later with \againframe{label}
%
%     plain
%			suppresses header, footer, and sidebar.
%
%     squeeze
%			minimizes vertical spacings as far as possible to make more content possible
%			on a slide
%
% 
%===================================================================================
%===================================================================================




\begin{document}


\begin{frame}{}
	\titlepage
\end{frame}


\begin{frame}{Outline}\label{tableofcontents}
   \tableofcontents
\end{frame}






%=======================================
\section{Section Name}
%=======================================

\begin{frame}
	\frametitle{Test Frametitle}
	\framesubtitle{Test yo}

    \begin{itemize}
		\item Test
		\item Test 2
		\item Test 3
    \end{itemize}

    \begin{description}
		\item[${G_3}'$:] Text goes here.
		\item WTF
		\item [Item Name] Description
    \end{description}
\end{frame}









%=======================================
\section{Blocks}
%=======================================


\begin{frame}
	\frametitle{blocks}

    \begin{block}{simple block title}
        Simple block text
    \end{block}
    
    \begin{exampleblock}{example block title}
         example block text
    \end{exampleblock}
   
    \begin{alertblock}{alert block title}
         alert block text
    \end{alertblock}
    
\end{frame}












%=========================================================
\section{Proof, Definitions, Lemmata, Remarks}
%=========================================================


\begin{frame}[fragile] % need fragile for verbatim
	\frametitle{Proofs etc}

    \begin{proof}
        Proof
    \end{proof}
    
    \begin{lemma}[XY -- A dual zu YX]
        Lemma
    \end{lemma}
    
    \begin{theorem}[T -- after Tarski]
        Theorem
    \end{theorem}
    
     \begin{rem}
		remark: first set
		  \begin{verbatim}
		    \newtheorem*{rem}{Remark}
		  \end{verbatim}
		  in preamble! 
     \end{rem}
\end{frame}








%=========================================================
\section{Overlays}
%=========================================================


\begin{frame}
	\frametitle{Overlays}
   \begin{itemize}
        \item Start
        \item<2-> so it follows
        \item<3-> then this
        \item<4-> then that
   \end{itemize}
\end{frame}



\begin{frame}{Overlays 2}
	\begin{itemize}
		\item<1> This is on the first only
		\item<-3> This is on the first three slides
		\item<2-4,6> This is on the second to fourth slides and the sixth slide
	\end{itemize}
\end{frame}



\begin{frame}[fragile]{Overlays shortcuts} % fragile for \verb
	\begin{itemize}
		\item<+-> This is on the first and all following slides
		\item<+-> This is on the second and all following slides
		\item<+-> This is on the third and all following slides
		\item<.-> This is the same as the last called \verb|<+->|, i.e. the last \verb|+|
	\end{itemize}
\end{frame}






\begin{frame}{Overlays shortcuts 2}
	\begin{itemize}[<+->]
		\item This is on the first and all following slides
		\item This is on the second and all following slides
		\item This is on the third and all following slides
		\item<1-> This is on the first and all following slides. You can override shortcuts
	\end{itemize}
\end{frame}









%=========================================================
\section{Two Columns}
%=========================================================

\begin{frame}
	\frametitle{Two column stuff}
    \begin{columns}
         \column{.55\textwidth}
                 \pgfimage[width=\textwidth]{figures/test.jpg}
         \column{.45\textwidth}
                 \begin{enumerate}
                 \item Start
                 \item Stop
                 \end{enumerate}
    \end{columns}
\end{frame}
















%=========================================================
\section{Images}
%=========================================================





%=======================================
\subsection{Two images}
%=======================================


\begin{frame}{Two images}
\begin{figure}[!htb]
	\centering
	\minipage{0.4\textwidth}
		\fbox{\includegraphics[height=3cm, keepaspectratio]{figures/test.jpg}}%
		\caption{
			I really don't 
			\label{fig:noidea1}
		}%
	\endminipage\hspace{1cm}   
	%
	\minipage{0.4\textwidth}
		\fbox{\includegraphics[height=3cm, keepaspectratio]{figures/test.jpg}}%
		\caption{
			\smash{indeed I don't} 
			\label{fig:noidea2}
		}%
	\endminipage
\end{figure}
\end{frame}










%====================================
\subsection{Full Page Image}
%====================================




{
	\setbeamertemplate{background}
	{
		\vbox to \paperheight{
			\vfil\hbox to \paperwidth{
				\hfil\includegraphics[width=\paperwidth]{figures/test.jpg}
			\hfil}
		}
	}
	\begin{frame}[plain]
		\frametitle{Full Page Image}
	\end{frame}
}





\begin{frame}[fragile]
	\frametitle{Small caption for big image}
	\begin{figure}[!htb]
		\begin{center}
			\includegraphics[height=6cm, keepaspectratio]{figures/musr_general_principle.png}%
			\caption*{  \setlength{\baselineskip}{6pt}
				{\tiny 
					Dalmas de Réotier, Pierre (2010): \textit{ 						
						Introduction to muon spin rotation and relaxation.
					}
					[Online]. Availible: \url{
						http://inac.cea.fr/Pisp/pierre.dalmas-de-reotier/introduction_muSR.pdf
					}
				}
			}%
		\end{center}
	\end{figure}          
\end{frame}






















%====================================
\subsection{Math}
%====================================




\begin{frame}
	\frametitle{Math}
	\begin{align}
			 f(z) &= 
			 	\lim\limits_{x\rightarrow \infty} \frac{\sin x}{x} = 0\\
		 	 \binom{a}{n} &= 
		 	 	\frac{a!}{(a-n)! n!}\\
		 	 \nonumber \int(z) dz &=  
		 	 	\frac14 \left[ \int\frac{e^{ia(u + 1)}}{u} du - \int\frac{e^{ia(u + 1)}}{u + 2}du   \right]\\[1em]
		 	 & \overset{z = 1 \Rightarrow u = 0}=  
		 	 	\frac{e^{i a}}{4} \left[\vphantom{ \int\limits_{\pi}^0} \smash{ \underbrace{\frac{\overbrace{e^{ia \epsilon e^{i \varphi}}}^{\rightarrow 1}} {\epsilon e^{i \varphi}} i \epsilon e^{i \varphi}}_{\rightarrow i}  d \varphi            - \int\limits_{\pi}^0 \underbrace{\frac{\overbrace{e^{ia \epsilon e^{i \varphi}}}^{\rightarrow 1}} {\underbrace{\epsilon e^{i \varphi}}_{\rightarrow 0} + 2} \underbrace{i \epsilon e^{i \varphi}}_{\rightarrow 0}}_{\rightarrow 0}  d \varphi  }\right]\\[2em]
		 	 %	
		 	 %
		 	 2 + 2 &= 4 \ \text{some more space after this line please.}\\[4em]\nonumber
	\end{align}
\end{frame}













%=========================================================
\section{Citations and References}
%=========================================================


\begin{frame}{Citations and References}
    Citing: \cite{AHF}
    
    \renewcommand*{\bibfont}{\footnotesize}
    \bibliography{references}
\end{frame}






\end{document}
