%preamble
\documentclass[11pt]{beamer}


\usepackage{lmodern} 		% Diese beiden packages sorgen für echte 
\usepackage[T1]{fontenc}	% Umlaute.

\usepackage{amssymb, amsmath, color, graphicx, float, setspace, tipa}
\usepackage[utf8]{inputenc} 
\usepackage[english]{babel}
\usepackage[justification=centering]{caption}
\addto\captionsenglish{\renewcommand{\figurename}{}} % Redefine how figures are called


%\usepackage[pdfpagelabels,pdfstartview = FitH,bookmarksopen = true,bookmarksnumbered = true,linkcolor = black,plainpages = false,hypertexnames = false,citecolor = black, breaklinks]{hyperref}
%\usepackage{url}
\usepackage{longtable} 		%Seitenübergreifende Tabelle. Vorlage siehe unten
\newtheorem*{bem}{Bemerkung} % Neue Theorem-Umgebung: Bemerkung
\newcommand{\fillframe}{\vskip0pt plus 1filll} 


% CAPTIONS
\usepackage[]{caption}
\addto\captionsenglish{\renewcommand{\figurename}{}} %Abbildungen nicht bzw. anders beschriften.
\captionsetup{font={small,sf}, labelfont=bf, format=plain}
% Use captionsetup below for no 'Figure' label
%\captionsetup{font={small,tt}, labelformat=empty, format=plain}



%===================
%BIBLIOGRAPHY
%===================
\usepackage[backend=bibtex,sorting=nyt,bibencoding=ascii,citestyle=authoryear]{biblatex}
\addbibresource{references.bib}
\usepackage{csquotes} %recommended when using babel and biblatex
%Abbreviations for Bibliography: http://adsabs.harvard.edu/abs_doc/aas_macros.html







%-----------------
% BEAMER-SPECIFIC
%-----------------

\usetheme{metropolis}
% deactivate a new page when a new section begins
\metroset{sectionpage=none} 
\usepackage{FiraSans}
\usefonttheme[onlymath]{serif}


% You can look at different options for usetheme, usecolortheme und usefonttheme 
% here: http://deic.uab.es/~iblanes/beamer_gallery/


% transparent Overlays (what's coming next on the slide is shown with reduced alpha)
%\setbeamercovered{transparent} 

% Removes navigation symbols at the bottom
%\beamertemplatenavigationsymbolsempty 

% Use page numbers as links
%\setbeamertemplate{footline}[frame]  
%    \setbeamertemplate{footline}{%
%    	\raisebox{5pt}{\makebox[\paperwidth]{\hfill\makebox[10pt]{\hyperlink{tableofcontents}{\scriptsize\insertframenumber}}}}}




%==============================================
% This file contains my definitions and
% newcommands. Makes things easy to copypaste
% between projects.
%==============================================


%--------------------------------------------
% Math Stuff
%--------------------------------------------

\newcommand{\corresponds}{\mathrel{\widehat{=}}}       % equals with hat

\newcommand {\arctanh}{\mathrm{arctanh}}               % Atanh
\newcommand{\arccot}{\mathrm{arccot }}                 % Acotanh

\newcommand{\e}{\mbox{e}}                              % e noncursive in math mode

\newcommand{\del}{\partial}                            % partial diff operator
\newcommand{\de}{\mathrm{d}}                           % differential d
\newcommand{\D}{\mathrm{d}}                            % differential d
\newcommand{\GRAD}{\mathrm{grad}\ }                    % gradient
\newcommand{\DIV}{\mathrm{div}\ }                      % divergence
\newcommand{\ROT}{\mathrm{rot}\ }                      % rotation

\newcommand{\CONST}{\mathrm{const.\ }}                 % constant
\newcommand{\var}{\mathrm{var}}                        % variance

\newcommand{\msol}{M_\odot}                            % solar mass
\newcommand{\order}{\mathcal{O}}                       % order, e.g. O(h^2)

\newcommand{\dete}{\mathrm{d}t}                        % dt
\newcommand{\delte}{\del t}                            % partial t
\newcommand{\dex}{\mathrm{d}x}                         % dx
\newcommand{\delx}{\del x}                             % partial x
\newcommand{\der}{\mathrm{d}r}                         % dr
\newcommand{\delr}{\del r}                             % partial r


\newcommand{\deldx}{\frac{\del}{\del x}}				% shortcut partial derivative, in line
\newcommand{\ddx}{\frac{\de}{\de x}}					% shortcut total derivative, in line
\newcommand{\DELDX}[1]{\frac{\del  #1}{\del x}}			% shortcut partial derivative, on fraction
\newcommand{\DDX}[1]{\frac{\de  #1}{\de x}}				% shortcut total derivative, on fraction

\newcommand{\deldvecx}{\frac{\del}{\del \x}}	   		% shortcut partial derivative, in line
\newcommand{\ddvecx}{\frac{\de}{\de \x}}				% shortcut total derivative, in line
\newcommand{\DELDVECX}[1]{\frac{\del  #1}{\del \x}}		% shortcut partial derivative, on fraction
\newcommand{\DDVECX}[1]{\frac{\de  #1}{\de \x}}			% shortcut total derivative, on fraction

\newcommand{\deldr}{\frac{\del}{\del r}}				% shortcut partial derivative, in line
\newcommand{\ddr}{\frac{\de}{\de r}}					% shortcut total derivative, in line
\newcommand{\DELDR}[1]{\frac{\del  #1}{\del r}}			% shortcut partial derivative, on fraction
\newcommand{\DDR}[1]{\frac{\de  #1}{\de r}}				% shortcut total derivative, on fraction


%------------------------------------------
%:Metainformationen

\title{Titel}
\author{Mladen Ivkovic\\
mladen.ivkovic@uzh.ch\\
}
\date{Datum}

%------------------------------------------


\begin{document}
%\pagestyle{plain}

%Titlepage
\maketitle
\clearpage
\tableofcontents %Auf englisch wechseln: Ändere usepackage ngeman babel in english babel
\clearpage

%Vorwort/Anmerkung
\cleardoublepage
\chapter*{Preface}
\markboth{Preface}{Preface}
\addcontentsline{toc}{chapter}{Preface}


Remove unless somebody else wants to write nice things about this work.


\lipsum[1-2]





%Section 1
\section{Kapitel 1}
\subsection{Unterkapitel 1.1}
\subsubsection{Unterunterkapitel 1.1.1}

%
\piccaption{Darstellung des Zahlenbereichs des Zweierkomplements mit acht Stellen\label{fig:tabelle_zweierkomplement}}
\parpic[r]{%
  \fbox{
    \includegraphics[width = 5.5cm, keepaspectratio]{images/tabelle_zweierkomplement.png}
  }
}
%
Die gängigste Form der Zahlensysteme sind Stellenwertsysteme. Eine Zahl $a$ wird in Form einer Reihe von Ziffern $z_i$ mit dazugehöriger Potenz der Basis $b^i$ dargestellt. Der Wert der Zahl ergibt sich dann als Summe der Werte aller Einzelstellen: $a = \sum\limits_{i}z_ib^i$.

\textbf{Umrechnung} in andere Zahlensysteme: Gegeben sei Zahl $Z$, umzuwandeln in System mit Basis $b$.
Eine angenehme Vorgehensweise gibt uns das \textbf{Horner Schema}\footnote{
Website mit Umrechnungen und Erklärungen: \url{http://www.arndt-bruenner.de/mathe/scripts/Zahlensysteme.htm}
}: Dividiere $Z$ durch $b$. Der Rest dieser Division ist die letzte Stelle der Zahl in der Basis $b$  (Einerstelle). Dividiere den Quotienten dieser Division wieder durch $b$. Der Rest dieser zweiten Division ergibt die zweite Stelle der Zahl in der neuen Basis. Wiederhole Divisionen, bis kein Rest mehr.






\section{Tabellen}
\subsection{Einfach}
\begin{center}
\begin{tabular}[c]{c | c | c || c| c | c || c | c || c | c | c || c| c| c}
\multicolumn{3}{c||}{Konjunktion}	&	\multicolumn{3}{c||}{Disjunktion} & \multicolumn{2}{c||}{Negation} & \multicolumn{3}{c||}{NAND} & \multicolumn{3}{c}{NOR}\\
\multicolumn{3}{c||}{UND}	&	\multicolumn{3}{c||}{ODER} & \multicolumn{2}{c||}{} & \multicolumn{3}{c||}{} & \multicolumn{3}{c}{}\\
\hline
$a$ & $b$ & $a$ $\wedge$ $b$ & $a$ & $b$ & $a$ $\vee$ $b$ & $a$ & $\bar{a}$ & $a$ & $b$ & $\overline{a \wedge b}$ & $a$ & $b$ & $\overline{a \vee b}$\\
\hline
0 & 0 & 0 & 0 & 0 & 0 & 0 & 1 & 0 & 0 & 1 & 0 & 0 & 1\\
0 & 1 & 0 & 0 & 1 & 1 & 1 & 0 & 0 & 1 & 1 & 0 & 1 & 0\\
1 & 0 & 0 & 1 & 0 & 1 & & & 1 & 0 & 1 & 1 & 0 & 0\\
1 & 1 & 1 & 1 & 1 & 1 & & & 1 & 1 & 0 & 1 & 1 & 0\\
\hline
\end{tabular}
\end{center}




%-----------------------------
\section{Zwei Bilder}

\begin{figure}[h!]
\centering
  \minipage{0.3\textwidth}
    \fbox{\includegraphics[height=2.5cm, keepaspectratio]{images/rsflipflop.png}}%
    \caption{RS-Flipflop}%
    \label{fig:rsflipflop}
  \endminipage\hspace{1cm}   
%
  \minipage{0.4\textwidth}
    \fbox{\includegraphics[height=2.5cm, keepaspectratio]{images/rsflipfloptakt.png}}%
    \caption{getaktetes RS-Flipflop}%
    \label{fig:rsflipfloptakt}
  \endminipage
\end{figure}

Dabei müssen wir eine Nebenbedingung $R \wedge S = 0$ setzen - $R$ und $S$ dürfen niemals gleichzeitig $= 1$ sein. In der Realisierung, dargestellt in Abb. \ref{fig:rsflipflop}, führt dies zu oszillationen. 

Will man ein taktgesteuertes RS-Flipflop, so braucht man lediglich das Taktsignal mit einem UND-Gatter jeweils mit dem $R$- und $S$-Eingang zu verbinden (siehe Abb. \ref{fig:rsflipfloptakt}).



\end{document}