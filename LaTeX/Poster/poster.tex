\documentclass{beamer}
%% Possible paper sizes: a0, a0b, a1, a2, a3, a4.
%% Possible orientations: portrait, landscape
%% Font sizes can be changed using the scale option.
\usepackage[size=a0,orientation=portrait,scale=1.1]{beamerposter}
\usetheme{LLT-poster}
% \usecolortheme{ComingClean}
% \usecolortheme{Entrepreneur}
% \usecolortheme{ConspicuousCreep}  %% VERY garish.
\usecolortheme{Durham}  %% VERY garish.

\usepackage[]{natbib}
\bibliographystyle{apalike}

\usepackage[UKenglish]{babel}


% \usepackage{wrapfig}
\usepackage{caption}
\captionsetup{labelformat=simple, font={footnotesize,sl}, labelfont={bf,footnotesize},
format=plain,labelsep=space}


% \usepackage{xspace}
% \newcommand{\swift}{\textsc{Swift}\xspace}
% \newcommand{\peano}{\textsc{Peano}\xspace}

\usepackage[utf8]{inputenc}
\usepackage[T1]{fontenc}
\usepackage{libertine}
\usepackage[scaled=0.92]{inconsolata}
\usepackage[libertine]{newtxmath}

\author[mladen.ivkovic@durham.ac.uk]{\textit{\textmd{Cristian Barrera-Hinojosa}}, \textmd{Mladen
Ivkovic}, \textit{\textmd{Pawel Radtke}}, \textit{\textmd{Tobias Weinzierl}}}
\title{SWIFT 2 : Keeping the Good, Discussing the Bad, Removing the Ugly}
% \subtitle{test} // DOESN'T WORK
\institute{Durham University}
% Optional foot image
\footimage{\includegraphics[height=4.55cm]{figures/logo-placeholder.png}\includegraphics[height=4.55cm]{figures/logo-placeholder.png}}














%=========================================================
%=========================================================

\begin{document}
\begin{frame}[fragile]\centering




%=========================================================
%=========================================================
\begin{block}{Long Block at Top}
    Lorem Ipsum
\end{block}




\begin{columns}[T]



%=========================================================
%=========================================================
%=========================================================
%=========================================================
%%%% First Column
\begin{column}{.48\textwidth}

\begin{block}{Overview}

Hic Forum Est

\end{block}

\end{column}






%%%% Second Column
%=========================================================
%=========================================================
%=========================================================
%=========================================================
\begin{column}{.48\textwidth}






\begin{block}{Our Goals}

\begin{itemize}
\item SWIFT's \textbf{task-based parallelism} has been proven successful. We aim to build and
expand on that in the context of particle based cosmological simulations.
\item SWIFT's engine needs to be replaced with a more flexible variant:
\begin{itemize}
 \item SWIFT 2 needs to be able to make effective use of \textbf{heterogeneous architectures}.
 \item Citations: \citet{weinzierlTwoParticleingridRealisations2016} \citet{reinarzExaHyPEEngineParallel2020} \citep{schayeFLAMINGOProjectCosmological2023}
\end{itemize}
\end{itemize}

\end{block}





% ==========================================================
% ==========================================================




\begin{block}{How Does It Work?}

\begin{minipage}{0.63\textwidth}


\begin{itemize}
\item While PEANO provides the adaptive mesh and mesh traversals, we need to tell it what physics
to solve.
\item We still use particles as discretisation elements. The adaptive mesh is
used to sort particles spatially and to allow for quick access to ``neighbouring'' particles.
\item Particles are stored using a ``dual tree'' management strategy
\citep{weinzierlTwoParticleingridRealisations2016}: One tree holds the particles
within the grid cells, while the other holds them within the grid vertices.
\end{itemize}
\vspace{.5em}


\end{minipage}\hfill
%
%
\begin{minipage}{0.35\textwidth}
%
\begin{figure}
\fbox{
\includegraphics[width=.95\linewidth]{figures/random_image.png}
}
\caption{The layers of PEANO.}
\label{fig:peano}
\end{figure}
%
\end{minipage}


\end{block}






\begin{block}{References}
   \renewcommand*{\bibfont}{\scriptsize}
   \bibliography{references}
\end{block}




\end{column}
\end{columns}


%=========================================================
%=========================================================
\begin{block}{Long Block at End}
    Lorem Ipsum
\end{block}



\end{frame}
\end{document}
